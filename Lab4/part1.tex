\newpage
\section*{Создание модуля для работы с матричными операциями и использование его в вычислении параметров линейной системы автоматического управления}

\subsection*{Создание модуля для работы с матрицами\\}

В основе модуля лежит класс \textit{Matrix}, который базируется на таком простом типе данных, как \textit{list}. Основная часть арифметических операций для данного класса была перегружена, для удобства представления операций в последующем. Было перегружено матричное умножение:

\paragraph{Транспонирование матрицы\\}

За транспонирование матрицы отвечает метод \textit(transpose()), который является внетренним методом класса \textit(Matrix):

\begin{lstlisting}
def transpose(self):
		return Matrix(list(map(list,zip(*self.__matrix))))
\end{lstlisting}
