\newpage

\section*{\centerline{Цель работы}}

Получение навыков работы с командной строкой UNIX и UNIX-подобных систем, а также навыки работы с файлами: просмотр редактирование, поиск и архивирование в GNU/Linux. Изучить основные команды и утилиты, поработать с правами на файлы и директории в GNU/LINUX. 
\newpage

\section*{\centerline{Выполнение}}
	\vspace{1cm}

	\subsection*{\centerline{Часть 1}}
		\vspace{0.5cm}
		\centerline{Цель задачи}
		1) Изучение команд для скачивания файлов из Интернета.\\
		2) Изучение команд для работы с архивами.\\
		3) Изучение команд для поиска файлов и слов в файлах.\\
		\vspace{1cm}


		\paragraph*{1.1)Скачивание файлов из интернета с использованием терминала\\\\}

		Команда \textit{wget --spider url} позволяет проверить доступность файла по указанному адресу url.\\
		\includegraphics [width=\textwidth]{picture1.png}\\
		\includegraphics [width=\textwidth]{picture2.png}\\
		\vspace{0.5cm}

		Команда \textit{wget -i filename} позволяет скачать файл по ссылке из файла. С помощью \textit{head -n 1} можно получить первую строку файла. Значок \textit{<} переводит вывод построчно в активный поток ввода\\
		\includegraphics [width=\textwidth]{picture3.png}\\
		\includegraphics [width=\textwidth]{picture4.png}\\
		\vspace{0.5cm}

		Скачивание файла по второй ссылке из файла. Также используется \textit{head}, но забирает он уже вторую строку\\
		\includegraphics [width=\textwidth]{wget(var1).png}\\

		Выборку остальных файлов можно осуществить при помощи команды \textit{sed} способной во входящем потоке отредактировать файл и удалить не нужные нам строки\\
		\includegraphics [width=\textwidth]{picture5.png}\\
		\vspace{0.5cm}

		Скачивание всех \textit{.jpeg}, \textit{.jpg} файлов с сайта про Лермонтова. Используется глубина рекурсии 1. Не создаются поддиректории.\\\
		\includegraphics [width=\textwidth]{picture6.png}\\
		\vspace{1cm}
		


		\paragraph*{1.2) Работа с архивами\\\\}

		Извлечение из \textit{.zip} архива файла Лермонтов.txt Используется утилита \textit{unzip}\\
		\includegraphics [width=\textwidth]{picture22.png}\\
		\vspace{0.5cm}

		Распаковка архивов \textit{.tar, .tar.gz, .tar.bz2} с выводом информации о процессе на экран (необходим ключик -v: verboose)\\
		\includegraphics [width=\textwidth]{tar(3).png}\\
		\vspace{0.5cm}


		\paragraph*{1.3) Поиск файлов, поиск по тексту\\\\}
		
		Нахождение всех файлов, созданных за последний 1 дней в домашней папке. Так как их было много, содержимое вывода не влезло в окно консоли\\
		\includegraphics [width=\textwidth]{picture23.png}\\
		\includegraphics [width=\textwidth]{picture12.png}\\
		\vspace{0.5cm}

		Нахождение всех файлов с фамилией Лермонтов с использованием команда \textit{find}, способной искать файлы по маске или регулярному выражению\\
		\includegraphics [width=\textwidth]{find_Lermontov.png}\\
		\vspace{0.5cm}

		Перемещение этого файла в папку \textit{Произведения Лермонтова}. Перемещение туда же файлов с картинками из сайта о Шекспире\\
		\includegraphics [width=\textwidth]{picture14.png}\\
		\includegraphics [width=\textwidth]{101.png}\\
		\includegraphics [width=\textwidth]{102.png}\\
		\vspace{0.5cm}

		Смена кодировки файла \textit{Лермонтов.txt} c \textit{windows-1251} на \textit{Unicode}. Используется команда 
		\textit{ivconv}\\
		\includegraphics [width=\textwidth]{picture16.png}\\
		\vspace{0.5cm}

		Поиск слов \textit{я, то?} (\textit{?}- является спец. символом) в файле, подсчет строк может быть осуществлен при использовании ключа \textit{-c}\\
		\includegraphics [width=\textwidth]{103.png}\\
		\includegraphics [width=\textwidth]{119.png}\\
		\vspace{0.5cm}

		Посчет количества строк в файлах с произведениями. Было решено использовать команду \textit{grep}, способную рпботать с регулярными выражениями\\
		\includegraphics [width=\textwidth]{picture17.png}\\
		\vspace{0.5cm}


		\paragraph*{1.4) Работа с архивами(архивирование)\\\\}

		Архивирование папок "Произведения Лермонтова" и "Картинки" в \textit{tar ,zip} и \textit{tar} сжатый \textit{gz}\\
		\includegraphics [width=\textwidth]{picture18.png}\\
		\includegraphics [width=\textwidth]{106.png}\\
		\includegraphics [width=\textwidth]{107.png}\\
		\vspace{0.5cm}

		Проверка архивирования. Проверяем содержимое архивов при помоши спец. ключа \textit{-t} для архивов tar и при помощи \textit{zipinfo} для zip \\
		\begin{center}
			\includegraphics [width=0.7\textwidth]{109.png}\\
			\includegraphics [width=\textwidth]{picture19.png}\\
		\end{center}
		\vspace{0.5cm}


