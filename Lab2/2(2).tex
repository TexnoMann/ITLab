\newpage
\section*{Часть 2}
	\paragraph{1)}
	Цель работы:  \\

	1) Изучение команд для просмотра типа файлов и содержимого файлов. \\

	2) Изучение команд для управления правами доступа к файлам и каталогам. \\

	3) Изучение команд для ввода/вывода данных. Перенаправление ввода/вывода. \\
	\subsection*{5 Просмотр содержимого файлов}
	
		\paragraph{1)}
	 	Для просмотра типа файла можно использовать несколько вариантов команд, 		одни из 			которых \textit{file <name>} и \textit{file} с ключом 			\textit{--mime-type} . Для 			поиска файлов в папке Произведения 				Лермонтова воспользуемся формой \textit{./				Произведения\ 				Лермонтова/*} , где * обозначает любой файл.
	\\
	\includegraphics[width=\textwidth]{110.png} %из Произведения Лермонтова%
	\\
	\includegraphics[width=\textwidth]{111.png} %из Lab2%
	\\
	
		\paragraph{2)}
		Для просмотра величины файла была выбрана команда \textit{ls -s <name>}, 		где ключ \textit{-s} означает size. \\
	Выбирать особо не из чего, поэтому был выбран файл \textit{Лермонтов.txt}.\\
	\includegraphics[width=\textwidth]{picture20.png} 
	\\
		Далее, чтобы вывести текст на экран без пустых строк использовалась команда 	\textit{cat -b <(grep ' ' <name>)}, где \textit{-b} отображает номер строк, 		стрелка потока обозначает последовательность действий, а операция в  				\textit{grep} не дает возможность отображению пустых страк, так как в них 			отсутствуют знаки пробела.\\
	\includegraphics[width=\textwidth]{picture21.png} 
	\\
	
		\paragraph{3)}
		За неимением других вариантов было выбрано произведение "Гроза". Для вывода 	первых 11 строк с помощью команды \textit{more} следует добавить \textit{-11}.
	\\
	\includegraphics[width=\textwidth]{112.png}
	\\
		\paragraph{4)}
		Для вывода текста стихотворения с использованием \textit{less} вводим 			\textit{less <name>}\\
	\includegraphics[width=\textwidth]{113(b).png}
	\\
	\includegraphics[width=\textwidth]{113(r).png}
	\\
		\paragraph{5)}
		В качестве анализа редакторов \textit{more} и \textit{less} можно 				предоствавить различия в просмотре документа: в дополнение к функциям 				\textit{more}  (постраничная или построчная прокрутка текста от начала до конца 	и поиск), команда less позволяет выполнять следующие операции: переход на 			указанную строку(number+g), переход в начало или  конец текста(g/G), прокрутка 		текста от конца к началу, поиск в обратном направлении.\\
		
	\subsection*{6 Управление правами доступа к файлам и каталогам}
	
		\paragraph{1)}
		Поскольку в 3 пункте 2 задания мы извлекали один файл, о нём и пойдет речь. 	Используем \textit{stat <name>} для вывода полной информации о файле.\\
	\includegraphics[width=\textwidth]{114.png}
	\\
	\includegraphics[width=\textwidth]{115.png}
	\\
		\paragraph{2)}
		Удаляем права на чтение в любимом произведение Лермонтова "Гроза", для 			этого вводим \textit{chmod a-r <name> }, где \textit{a-r} означает удаление(-) 		прав у всех (all)  пользователей на чтение (read). Поскольку прав на чтание 		нет, прочитать информацию из файла мы не можем :с
	Прав нет - сделать нельзя.\\
	\includegraphics[width=\textwidth]{116.png}
	\\
		\paragraph{3)}
		Берем "Грозу" и удаляем на нее права на запись. Чтобы это сделать пишем: 		\textit{chmod a-w}, где \textit{a-w} - удалить(-) у всех(all) право на 				запись(write). Проверяем попыткой заменить пробемы на ничего, то есть удалить 		пробелы в тексте произведения. Сделать не можем - значит работает.\\
	\includegraphics[width=\textwidth]{117.png}
	\\
	
		\paragraph{4)}
		Добавляем право выполнения членам группы и остальным  для документа 			Лермонтов.txt . Для этого используем команду \textit{chmod go+x}. Go(group and 		others) +(добавить) x(execute). Выполнить не получается, ну мы и не группа, и 		не остальные.\\
	\includegraphics[width=\textwidth]{118.png}
	\\
		\paragraph{5)}
		Сощдаём текстовый файл hello_world.sh с помощью редактора nano: 				\textit{nano}, прописываем текст $#!/bin/bash echo "Hello World"$, сохраняем и 		выходим. Выполняем проверку - всё работает, ехууу)) Проверяем путь - вё хорошо.		\\
	\includegraphics[width=\textwidth]{picture24.png}
	\\
	\includegraphics[width=\textwidth]{picture25.png}
	\\
		\paragraph{6)}
		Поочередно меняем значения записи/чтения/выполнения, для чего используем 		\textit{chmod} в различных конфигурациях.\\
	\includegraphics[width=0.7\textwidth]{picture26.png}
	\\	
	\includegraphics[width=\textwidth]{picture27.png}
	\\
		
	\vspace{2cm}	
	
	\begin{center}
		\textit{Спасибо за внимание:)}
	\end{center}

	  
	  
	  
		